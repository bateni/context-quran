\startbuffer[\q:12:5]
قَالَ یَٰبُنَیَّ لَا تَقۡصُصۡ رُءۡیَاكَ عَلَىٰۤ إِخۡوَتِكَ فَیَكِیدُوا۟ لَكَ كَیۡدًاۖ إِنَّ ٱلشَّیۡطَٰنَ لِلۡإِنسَٰنِ عَدُوࣱّ مُّبِینࣱ%
\stopbuffer%
\startbuffer[\q:12:6]
وَكَذَٰلِكَ یَجۡتَبِیكَ رَبُّكَ وَیُعَلِّمُكَ مِن تَأۡوِیلِ ٱلۡأَحَادِیثِ وَیُتِمُّ نِعۡمَتَهُۥ عَلَیۡكَ وَعَلَىٰۤ ءَالِ یَعۡقُوبَ كَمَاۤ أَتَمَّهَا عَلَىٰۤ أَبَوَیۡكَ مِن قَبۡلُ إِبۡرَٰهِیمَ وَإِسۡحَٰقَۚ إِنَّ رَبَّكَ عَلِیمٌ حَكِیمࣱ%
\stopbuffer%
\startbuffer[\q:12:7]
۞ لَّقَدۡ كَانَ فِی یُوسُفَ وَإِخۡوَتِهِۦۤ ءَایَٰتࣱ لِّلسَّاۤئِلِینَ%
\stopbuffer%
\startbuffer[\q:12:8]
إِذۡ قَالُوا۟ لَیُوسُفُ وَأَخُوهُ أَحَبُّ إِلَىٰۤ أَبِینَا مِنَّا وَنَحۡنُ عُصۡبَةٌ إِنَّ أَبَانَا لَفِی ضَلَٰلࣲ مُّبِینٍ%
\stopbuffer%
\startbuffer[\q:12:9]
ٱقۡتُلُوا۟ یُوسُفَ أَوِ ٱطۡرَحُوهُ أَرۡضࣰا یَخۡلُ لَكُمۡ وَجۡهُ أَبِیكُمۡ وَتَكُونُوا۟ مِنۢ بَعۡدِهِۦ قَوۡمࣰا صَٰلِحِینَ%
\stopbuffer%
\startbuffer[\q:12:10]
قَالَ قَاۤئِلࣱ مِّنۡهُمۡ لَا تَقۡتُلُوا۟ یُوسُفَ وَأَلۡقُوهُ فِی غَیَٰبَتِ ٱلۡجُبِّ یَلۡتَقِطۡهُ بَعۡضُ ٱلسَّیَّارَةِ إِن كُنتُمۡ فَٰعِلِینَ%
\stopbuffer%
\startbuffer[\q:12:11]
قَالُوا۟ یَٰۤأَبَانَا مَالَكَ لَا تَأۡمَـ۫نَّا عَلَىٰ یُوسُفَ وَإِنَّا لَهُۥ لَنَٰصِحُونَ%
\stopbuffer%
\startbuffer[\q:12:12]
أَرۡسِلۡهُ مَعَنَا غَدࣰا یَرۡتَعۡ وَیَلۡعَبۡ وَإِنَّا لَهُۥ لَحَٰفِظُونَ%
\stopbuffer%
\startbuffer[\q:12:13]
قَالَ إِنِّی لَیَحۡزُنُنِیۤ أَن تَذۡهَبُوا۟ بِهِۦ وَأَخَافُ أَن یَأۡكُلَهُ ٱلذِّئۡبُ وَأَنتُمۡ عَنۡهُ غَٰفِلُونَ%
\stopbuffer%
\startbuffer[\q:12:14]
قَالُوا۟ لَئِنۡ أَكَلَهُ ٱلذِّئۡبُ وَنَحۡنُ عُصۡبَةٌ إِنَّاۤ إِذࣰا لَّخَٰسِرُونَ%
\stopbuffer%
\startbuffer[\q:12:15]
فَلَمَّا ذَهَبُوا۟ بِهِۦ وَأَجۡمَعُوۤا۟ أَن یَجۡعَلُوهُ فِی غَیَٰبَتِ ٱلۡجُبِّۚ وَأَوۡحَیۡنَاۤ إِلَیۡهِ لَتُنَبِّئَنَّهُم بِأَمۡرِهِمۡ هَٰذَا وَهُمۡ لَا یَشۡعُرُونَ%
\stopbuffer%
\startbuffer[\q:12:16]
وَجَاۤءُوۤ أَبَاهُمۡ عِشَاۤءࣰ یَبۡكُونَ%
\stopbuffer%
\startbuffer[\q:12:17]
قَالُوا۟ یَٰۤأَبَانَاۤ إِنَّا ذَهَبۡنَا نَسۡتَبِقُ وَتَرَكۡنَا یُوسُفَ عِندَ مَتَٰعِنَا فَأَكَلَهُ ٱلذِّئۡبُۖ وَمَاۤ أَنتَ بِمُؤۡمِنࣲ لَّنَا وَلَوۡ كُنَّا صَٰدِقِینَ%
\stopbuffer%
\startbuffer[\q:12:18]
وَجَاۤءُو عَلَىٰ قَمِیصِهِۦ بِدَمࣲ كَذِبࣲۚ قَالَ بَلۡ سَوَّلَتۡ لَكُمۡ أَنفُسُكُمۡ أَمۡرࣰاۖ فَصَبۡرࣱ جَمِیلࣱۖ وَٱللَّهُ ٱلۡمُسۡتَعَانُ عَلَىٰ مَا تَصِفُونَ%
\stopbuffer%
\startbuffer[\q:12:19]
وَجَاۤءَتۡ سَیَّارَةࣱ فَأَرۡسَلُوا۟ وَارِدَهُمۡ فَأَدۡلَىٰ دَلۡوَهُۥۖ قَالَ یَٰبُشۡرَىٰ هَٰذَا غُلَٰمࣱۚ وَأَسَرُّوهُ بِضَٰعَةࣰۚ وَٱللَّهُ عَلِیمُۢ بِمَا یَعۡمَلُونَ%
\stopbuffer%
\startbuffer[\q:12:20]
وَشَرَوۡهُ بِثَمَنِۭ بَخۡسࣲ دَرَٰهِمَ مَعۡدُودَةࣲ وَكَانُوا۟ فِیهِ مِنَ ٱلزَّٰهِدِینَ%
\stopbuffer%
\startbuffer[\q:12:21]
وَقَالَ ٱلَّذِی ٱشۡتَرَىٰهُ مِن مِّصۡرَ لِٱمۡرَأَتِهِۦۤ أَكۡرِمِی مَثۡوَىٰهُ عَسَىٰۤ أَن یَنفَعَنَاۤ أَوۡ نَتَّخِذَهُۥ وَلَدࣰاۚ وَكَذَٰلِكَ مَكَّنَّا لِیُوسُفَ فِی ٱلۡأَرۡضِ وَلِنُعَلِّمَهُۥ مِن تَأۡوِیلِ ٱلۡأَحَادِیثِۚ وَٱللَّهُ غَالِبٌ عَلَىٰۤ أَمۡرِهِۦ وَلَٰكِنَّ أَكۡثَرَ ٱلنَّاسِ لَا یَعۡلَمُونَ%
\stopbuffer%
\startbuffer[\q:12:22]
وَلَمَّا بَلَغَ أَشُدَّهُۥۤ ءَاتَیۡنَٰهُ حُكۡمࣰا وَعِلۡمࣰاۚ وَكَذَٰلِكَ نَجۡزِی ٱلۡمُحۡسِنِینَ%
\stopbuffer%
\startbuffer[\q:12:23]
وَرَٰوَدَتۡهُ ٱلَّتِی هُوَ فِی بَیۡتِهَا عَن نَّفۡسِهِۦ وَغَلَّقَتِ ٱلۡأَبۡوَٰبَ وَقَالَتۡ هَیۡتَ لَكَۚ قَالَ مَعَاذَ ٱللَّهِۖ إِنَّهُۥ رَبِّیۤ أَحۡسَنَ مَثۡوَایَۖ إِنَّهُۥ لَا یُفۡلِحُ ٱلظَّٰلِمُونَ%
\stopbuffer%
\startbuffer[\q:12:24]
وَلَقَدۡ هَمَّتۡ بِهِۦۖ وَهَمَّ بِهَا لَوۡلَاۤ أَن رَّءَا بُرۡهَٰنَ رَبِّهِۦۚ كَذَٰلِكَ لِنَصۡرِفَ عَنۡهُ ٱلسُّوۤءَ وَٱلۡفَحۡشَاۤءَۚ إِنَّهُۥ مِنۡ عِبَادِنَا ٱلۡمُخۡلَصِینَ%
\stopbuffer%
\startbuffer[\q:12:25]
وَٱسۡتَبَقَا ٱلۡبَابَ وَقَدَّتۡ قَمِیصَهُۥ مِن دُبُرࣲ وَأَلۡفَیَا سَیِّدَهَا لَدَا ٱلۡبَابِۚ قَالَتۡ مَا جَزَاۤءُ مَنۡ أَرَادَ بِأَهۡلِكَ سُوۤءًا إِلَّاۤ أَن یُسۡجَنَ أَوۡ عَذَابٌ أَلِیمࣱ%
\stopbuffer%
\startbuffer[\q:12:26]
قَالَ هِیَ رَٰوَدَتۡنِی عَن نَّفۡسِیۚ وَشَهِدَ شَاهِدࣱ مِّنۡ أَهۡلِهَاۤ إِن كَانَ قَمِیصُهُۥ قُدَّ مِن قُبُلࣲ فَصَدَقَتۡ وَهُوَ مِنَ ٱلۡكَٰذِبِینَ%
\stopbuffer%
\startbuffer[\q:12:27]
وَإِن كَانَ قَمِیصُهُۥ قُدَّ مِن دُبُرࣲ فَكَذَبَتۡ وَهُوَ مِنَ ٱلصَّٰدِقِینَ%
\stopbuffer%
\startbuffer[\q:12:28]
فَلَمَّا رَءَا قَمِیصَهُۥ قُدَّ مِن دُبُرࣲ قَالَ إِنَّهُۥ مِن كَیۡدِكُنَّۖ إِنَّ كَیۡدَكُنَّ عَظِیمࣱ%
\stopbuffer%
\startbuffer[\q:12:29]
یُوسُفُ أَعۡرِضۡ عَنۡ هَٰذَاۚ وَٱسۡتَغۡفِرِی لِذَنۢبِكِۖ إِنَّكِ كُنتِ مِنَ ٱلۡخَاطِءِینَ%
\stopbuffer%
\startbuffer[\q:12:30]
۞ وَقَالَ نِسۡوَةࣱ فِی ٱلۡمَدِینَةِ ٱمۡرَأَتُ ٱلۡعَزِیزِ تُرَٰوِدُ فَتَىٰهَا عَن نَّفۡسِهِۦۖ قَدۡ شَغَفَهَا حُبًّاۖ إِنَّا لَنَرَىٰهَا فِی ضَلَٰلࣲ مُّبِینࣲ%
\stopbuffer%
\startbuffer[\q:12:31]
فَلَمَّا سَمِعَتۡ بِمَكۡرِهِنَّ أَرۡسَلَتۡ إِلَیۡهِنَّ وَأَعۡتَدَتۡ لَهُنَّ مُتَّكَءࣰا وَءَاتَتۡ كُلَّ وَٰحِدَةࣲ مِّنۡهُنَّ سِكِّینࣰا وَقَالَتِ ٱخۡرُجۡ عَلَیۡهِنَّۖ فَلَمَّا رَأَیۡنَهُۥۤ أَكۡبَرۡنَهُۥ وَقَطَّعۡنَ أَیۡدِیَهُنَّ وَقُلۡنَ حَٰشَ لِلَّهِ مَا هَٰذَا بَشَرًا إِنۡ هَٰذَاۤ إِلَّا مَلَكࣱ كَرِیمࣱ%
\stopbuffer%
\startbuffer[\q:12:32]
قَالَتۡ فَذَٰلِكُنَّ ٱلَّذِی لُمۡتُنَّنِی فِیهِۖ وَلَقَدۡ رَٰوَدتُّهُۥ عَن نَّفۡسِهِۦ فَٱسۡتَعۡصَمَۖ وَلَئِن لَّمۡ یَفۡعَلۡ مَاۤ ءَامُرُهُۥ لَیُسۡجَنَنَّ وَلَیَكُونࣰا مِّنَ ٱلصَّٰغِرِینَ%
\stopbuffer%
\startbuffer[\q:12:33]
قَالَ رَبِّ ٱلسِّجۡنُ أَحَبُّ إِلَیَّ مِمَّا یَدۡعُونَنِیۤ إِلَیۡهِۖ وَإِلَّا تَصۡرِفۡ عَنِّی كَیۡدَهُنَّ أَصۡبُ إِلَیۡهِنَّ وَأَكُن مِّنَ ٱلۡجَٰهِلِینَ%
\stopbuffer%
\startbuffer[\q:12:34]
فَٱسۡتَجَابَ لَهُۥ رَبُّهُۥ فَصَرَفَ عَنۡهُ كَیۡدَهُنَّۚ إِنَّهُۥ هُوَ ٱلسَّمِیعُ ٱلۡعَلِیمُ%
\stopbuffer%
\startbuffer[\q:12:35]
ثُمَّ بَدَا لَهُم مِّنۢ بَعۡدِ مَا رَأَوُا۟ ٱلۡءَایَٰتِ لَیَسۡجُنُنَّهُۥ حَتَّىٰ حِینࣲ%
\stopbuffer%
\startbuffer[\q:12:36]
وَدَخَلَ مَعَهُ ٱلسِّجۡنَ فَتَیَانِۖ قَالَ أَحَدُهُمَاۤ إِنِّیۤ أَرَىٰنِیۤ أَعۡصِرُ خَمۡرࣰاۖ وَقَالَ ٱلۡءَاخَرُ إِنِّیۤ أَرَىٰنِیۤ أَحۡمِلُ فَوۡقَ رَأۡسِی خُبۡزࣰا تَأۡكُلُ ٱلطَّیۡرُ مِنۡهُۖ نَبِّئۡنَا بِتَأۡوِیلِهِۦۤۖ إِنَّا نَرَىٰكَ مِنَ ٱلۡمُحۡسِنِینَ%
\stopbuffer%
\startbuffer[\q:12:37]
قَالَ لَا یَأۡتِیكُمَا طَعَامࣱ تُرۡزَقَانِهِۦۤ إِلَّا نَبَّأۡتُكُمَا بِتَأۡوِیلِهِۦ قَبۡلَ أَن یَأۡتِیَكُمَاۚ ذَٰلِكُمَا مِمَّا عَلَّمَنِی رَبِّیۤۚ إِنِّی تَرَكۡتُ مِلَّةَ قَوۡمࣲ لَّا یُؤۡمِنُونَ بِٱللَّهِ وَهُم بِٱلۡءَاخِرَةِ هُمۡ كَٰفِرُونَ%
\stopbuffer%
\startbuffer[\q:12:38]
وَٱتَّبَعۡتُ مِلَّةَ ءَابَاۤءِیۤ إِبۡرَٰهِیمَ وَإِسۡحَٰقَ وَیَعۡقُوبَۚ مَا كَانَ لَنَاۤ أَن نُّشۡرِكَ بِٱللَّهِ مِن شَیۡءࣲۚ ذَٰلِكَ مِن فَضۡلِ ٱللَّهِ عَلَیۡنَا وَعَلَى ٱلنَّاسِ وَلَٰكِنَّ أَكۡثَرَ ٱلنَّاسِ لَا یَشۡكُرُونَ%
\stopbuffer%
\startbuffer[\q:12:39]
یَٰصَٰحِبَیِ ٱلسِّجۡنِ ءَأَرۡبَابࣱ مُّتَفَرِّقُونَ خَیۡرٌ أَمِ ٱللَّهُ ٱلۡوَٰحِدُ ٱلۡقَهَّارُ%
\stopbuffer%
\startbuffer[\q:12:40]
مَا تَعۡبُدُونَ مِن دُونِهِۦۤ إِلَّاۤ أَسۡمَاۤءࣰ سَمَّیۡتُمُوهَاۤ أَنتُمۡ وَءَابَاۤؤُكُم مَّاۤ أَنزَلَ ٱللَّهُ بِهَا مِن سُلۡطَٰنٍۚ إِنِ ٱلۡحُكۡمُ إِلَّا لِلَّهِ أَمَرَ أَلَّا تَعۡبُدُوۤا۟ إِلَّاۤ إِیَّاهُۚ ذَٰلِكَ ٱلدِّینُ ٱلۡقَیِّمُ وَلَٰكِنَّ أَكۡثَرَ ٱلنَّاسِ لَا یَعۡلَمُونَ%
\stopbuffer%
\startbuffer[\q:12:41]
یَٰصَٰحِبَیِ ٱلسِّجۡنِ أَمَّاۤ أَحَدُكُمَا فَیَسۡقِی رَبَّهُۥ خَمۡرࣰاۖ وَأَمَّا ٱلۡءَاخَرُ فَیُصۡلَبُ فَتَأۡكُلُ ٱلطَّیۡرُ مِن رَّأۡسِهِۦۚ قُضِیَ ٱلۡأَمۡرُ ٱلَّذِی فِیهِ تَسۡتَفۡتِیَانِ%
\stopbuffer%
\startbuffer[\q:12:42]
وَقَالَ لِلَّذِی ظَنَّ أَنَّهُۥ نَاجࣲ مِّنۡهُمَا ٱذۡكُرۡنِی عِندَ رَبِّكَ فَأَنسَىٰهُ ٱلشَّیۡطَٰنُ ذِكۡرَ رَبِّهِۦ فَلَبِثَ فِی ٱلسِّجۡنِ بِضۡعَ سِنِینَ%
\stopbuffer%
\startbuffer[\q:12:43]
وَقَالَ ٱلۡمَلِكُ إِنِّیۤ أَرَىٰ سَبۡعَ بَقَرَٰتࣲ سِمَانࣲ یَأۡكُلُهُنَّ سَبۡعٌ عِجَافࣱ وَسَبۡعَ سُنۢبُلَٰتٍ خُضۡرࣲ وَأُخَرَ یَابِسَٰتࣲۖ یَٰۤأَیُّهَا ٱلۡمَلَأُ أَفۡتُونِی فِی رُءۡیَٰیَ إِن كُنتُمۡ لِلرُّءۡیَا تَعۡبُرُونَ%
\stopbuffer%
\startbuffer[\q:12:44]
قَالُوۤا۟ أَضۡغَٰثُ أَحۡلَٰمࣲۖ وَمَا نَحۡنُ بِتَأۡوِیلِ ٱلۡأَحۡلَٰمِ بِعَٰلِمِینَ%
\stopbuffer%
\startbuffer[\q:12:45]
وَقَالَ ٱلَّذِی نَجَا مِنۡهُمَا وَٱدَّكَرَ بَعۡدَ أُمَّةٍ أَنَا۠ أُنَبِّئُكُم بِتَأۡوِیلِهِۦ فَأَرۡسِلُونِ%
\stopbuffer%
\startbuffer[\q:12:46]
یُوسُفُ أَیُّهَا ٱلصِّدِّیقُ أَفۡتِنَا فِی سَبۡعِ بَقَرَٰتࣲ سِمَانࣲ یَأۡكُلُهُنَّ سَبۡعٌ عِجَافࣱ وَسَبۡعِ سُنۢبُلَٰتٍ خُضۡرࣲ وَأُخَرَ یَابِسَٰتࣲ لَّعَلِّیۤ أَرۡجِعُ إِلَى ٱلنَّاسِ لَعَلَّهُمۡ یَعۡلَمُونَ%
\stopbuffer%
\startbuffer[\q:12:47]
قَالَ تَزۡرَعُونَ سَبۡعَ سِنِینَ دَأَبࣰا فَمَا حَصَدتُّمۡ فَذَرُوهُ فِی سُنۢبُلِهِۦۤ إِلَّا قَلِیلࣰا مِّمَّا تَأۡكُلُونَ%
\stopbuffer%
\startbuffer[\q:12:48]
ثُمَّ یَأۡتِی مِنۢ بَعۡدِ ذَٰلِكَ سَبۡعࣱ شِدَادࣱ یَأۡكُلۡنَ مَا قَدَّمۡتُمۡ لَهُنَّ إِلَّا قَلِیلࣰا مِّمَّا تُحۡصِنُونَ%
\stopbuffer%
\startbuffer[\q:12:49]
ثُمَّ یَأۡتِی مِنۢ بَعۡدِ ذَٰلِكَ عَامࣱ فِیهِ یُغَاثُ ٱلنَّاسُ وَفِیهِ یَعۡصِرُونَ%
\stopbuffer%
\startbuffer[\q:12:50]
وَقَالَ ٱلۡمَلِكُ ٱئۡتُونِی بِهِۦۖ فَلَمَّا جَاۤءَهُ ٱلرَّسُولُ قَالَ ٱرۡجِعۡ إِلَىٰ رَبِّكَ فَسۡءَلۡهُ مَا بَالُ ٱلنِّسۡوَةِ ٱلَّٰتِی قَطَّعۡنَ أَیۡدِیَهُنَّۚ إِنَّ رَبِّی بِكَیۡدِهِنَّ عَلِیمࣱ%
\stopbuffer%
\startbuffer[\q:12:51]
قَالَ مَا خَطۡبُكُنَّ إِذۡ رَٰوَدتُّنَّ یُوسُفَ عَن نَّفۡسِهِۦۚ قُلۡنَ حَٰشَ لِلَّهِ مَا عَلِمۡنَا عَلَیۡهِ مِن سُوۤءࣲۚ قَالَتِ ٱمۡرَأَتُ ٱلۡعَزِیزِ ٱلۡءَٰنَ حَصۡحَصَ ٱلۡحَقُّ أَنَا۠ رَٰوَدتُّهُۥ عَن نَّفۡسِهِۦ وَإِنَّهُۥ لَمِنَ ٱلصَّٰدِقِینَ%
\stopbuffer%
\startbuffer[\q:12:52]
ذَٰلِكَ لِیَعۡلَمَ أَنِّی لَمۡ أَخُنۡهُ بِٱلۡغَیۡبِ وَأَنَّ ٱللَّهَ لَا یَهۡدِی كَیۡدَ ٱلۡخَاۤئِنِینَ%
\stopbuffer%
\startbuffer[\q:12:53]
۞ وَمَاۤ أُبَرِّئُ نَفۡسِیۤۚ إِنَّ ٱلنَّفۡسَ لَأَمَّارَةُۢ بِٱلسُّوۤءِ إِلَّا مَا رَحِمَ رَبِّیۤۚ إِنَّ رَبِّی غَفُورࣱ رَّحِیمࣱ%
\stopbuffer%
\startbuffer[\q:12:54]
وَقَالَ ٱلۡمَلِكُ ٱئۡتُونِی بِهِۦۤ أَسۡتَخۡلِصۡهُ لِنَفۡسِیۖ فَلَمَّا كَلَّمَهُۥ قَالَ إِنَّكَ ٱلۡیَوۡمَ لَدَیۡنَا مَكِینٌ أَمِینࣱ%
\stopbuffer%
